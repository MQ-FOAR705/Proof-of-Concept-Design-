\documentclass[a4paper,12pt]{article}
\usepackage[english]{babel}
\usepackage[utf8]{inputenc}

%
% For alternative styles, see the biblatex manual:
% http://mirrors.ctan.org/macros/latex/contrib/biblatex/doc/biblatex.pdf
%
% The 'verbose' family of styles produces full citations in footnotes, 
% with and a variety of options for ibidem abbreviations.
%
\usepackage{csquotes}
\usepackage[style=verbose-ibid,backend=bibtex]{biblatex}
\bibliography{sample}

\usepackage{lipsum} % for dummy text

\title{Proof of Concept (Design)}

\author{John Hundley}

\date{\today}

\begin{document}
\maketitle

\section{The User Story}

\begin{itemize}


\item{ 1. As a student, I want to perform a single search across multiple documents, so that I can procure a specific theme or term.}


\item{2. As a student, I want a program to generate bibliographic details for me, so that I can ensure these details are correct and save vast amounts of time.}


\item{3. As a student, I want to annotate my bibliography in a separate 'meta' document, so that it is stored safely, securely, and separately.}

\end{itemize} 


\section{Acceptance Criteria}

\begin{itemize} 

\item{1. As a student, I want to perform a single search across multiple documents by: 


1) Uploading required sources in Voyant.


2) Search key term in search engine.


3) Analyse areas of correlation, word count, frequency, and distribution across the sources by using tools provided by Voyant (trends, cirrus, frequency, etc. }


\item {2. As a student, I want a program to generate bibliographic details for me by:


1) Downloading Zotero and Zotero Connector.


2) Accessing source through online database.


3) Clicking 'Save to Zotero' option embedded in Chrome toolbar.


4) Open and check Zotero to ensure the source's bibliographic details are saved correctly.

5) Select source and right-click 'create biliography from item'.

6) Select desired format.}

\item{As a student, I want to annotate my bibliography in a separate 'meta' document by:

1) Opening Zotero. 

2) Selecting source.

3) Clicking 'Add notes'.

4) Entering notes.}

\end{itemize}

\section{Prerequisites}


\begin{itemize}
   
   
\item As a student, to perform a single search across multiple documents will require Voyant to: 1) accept the format of files I upload 2) search through the files with precision and accuracy 3) return the search with information pertaining to word count, frequency, distribution, and so on. 


\item As a student, to have a program generate bibliographic details for me will require Zotero: 1) to be downloaded and installed on the Chrome toolbar 2) checked to see if the correct formatting style is available (Harvard, Chicago, etc.) 3) the creation of the bibliographic entry will be available to me via html or RTF. 
    
    
\item As a student, to annotate my bibliography in a separate meta-document will require Zotero: 1) to permit me to generate notes on a chosen document. 

\end{itemize}


\section{Themes}


The main theme across these areas is time. My user stories all focus on an aspect of my role as a student and researcher that is time-consuming and can be readily automated. 


\begin{itemize}


\item 1. Performing a single search across multiple documents for a certain theme or term saves me time. It also provides a level of accuracy in identifying areas of correlation that I possibly could not match. 

\item 2. Using a program to generate bibliographic details firstly, ensures that sources are referencing correctly and secondly, saves a great deal of time referencing them manually. 

\item 3. Annotating my bibliography in a meta document provides a safe place to store my notes and bibliography. This also saves time when looking to use a specific source. 

    
\end{itemize}


\section{Quality Assurance}


\begin{itemize} 


\item The above prerequisites can all be tested in a simple manner. Most of these tests were completed in the Elaboration stage of this project's development. They can be tested in a basic way by using the above acceptance criteria. 


\end{itemize} 


\end{document}