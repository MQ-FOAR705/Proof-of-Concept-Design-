\documentclass[a4paper,12pt]{article}
\usepackage[english]{babel}
\usepackage[utf8]{inputenc}

%
% For alternative styles, see the biblatex manual:
% http://mirrors.ctan.org/macros/latex/contrib/biblatex/doc/biblatex.pdf
%
% The 'verbose' family of styles produces full citations in footnotes, 
% with and a variety of options for ibidem abbreviations.
%
\usepackage{csquotes}
\usepackage[style=verbose-ibid,backend=bibtex]{biblatex}
\bibliography{sample}

\usepackage{lipsum} % for dummy text

\title{Proof of Concept (Design)}

\author{John Hundley}

\date{\today}

\begin{document}
\maketitle

\section{The User Story}

\begin{itemize}


\item{ 1. As a student, I want to perform a single search across multiple documents, so that I can procure a specific theme or term.}


\item{2. As a student, I want a program to generate bibliographic details for me, so that I can ensure these details are correct and save vast amounts of time.}


\item{3. As a student, I want to annotate my bibliography in a separate 'meta' document, so that it is stored safely, securely, and separately.}

\end{itemize} 


\section{Acceptance Criteria}

\begin{itemize} 

\item{1. As a student, I want to perform a single search across multiple documents by: 


1) Uploading required sources in Voyant.


2) Search key term in search engine.


3) Analyse areas of correlation, word count, frequency, and distribution across the sources by using tools provided by Voyant (trends, cirrus, frequency, etc. }


\item {2. As a student, I want a program to generate bibliographic details for me by:


1) Downloading Zotero and Zotero Connector.


2) Accessing source through online database.


3) Clicking 'Save to Zotero' option embedded in Chrome toolbar.


4) Open and check Zotero to ensure the source's bibliographic details are saved correctly.

5) Select source and right-click 'create biliography from item'.

6) Select desired format.}

\item{As a student, I want to annotate my bibliography in a separate 'meta' document by:

1) Opening Zotero. 

2) Selecting source.

3) Clicking 'Add notes'.

4) Entering notes.}

\end{itemize}

\section{Themes}

\end{document}